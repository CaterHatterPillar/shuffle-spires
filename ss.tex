\documentclass[11pt,twocolumn]{article}

\usepackage{array}
\usepackage{subfig}
\usepackage{hyperref}
\usepackage{geometry}
\usepackage{graphicx}
\usepackage{booktabs}
\usepackage{paralist}
\usepackage{verbatim}
\usepackage{enumitem}

\geometry{a4paper}
\usepackage{fancyhdr}

\usepackage[utf8]{inputenc}

\pagestyle{fancy}
\renewcommand{\headrulewidth}{0pt}
\lhead{}\chead{}\rhead{}
\lfoot{}\cfoot{\thepage}\rfoot{}
\usepackage{sectsty}
\allsectionsfont{\sffamily\mdseries\upshape}
\setcounter{secnumdepth}{-1} % This is a tad obscure, and the past me didn't document as to why this is here.

\title{Shuffle-Spires}
\author{Eric Nilsson}
\date{}
\begin{document}
\maketitle

% Introduction
\noindent
\textbf{Shuffle-Spires} is a chaotic card-/dice-/board-game designed around four players, using only a standard deck of cards and a ten-sided die.
All you need in order to play \textbf{Shuffle-Spires} is:
\begin{itemize}[noitemsep]
\renewcommand{\labelitemi}{$\bullet$}
\item a set of playing cards
\item one of those dice* mentioned earlier
\item this rule-set
\item four players
\end{itemize}

% ss_story.tex

% Legend of the Shuffling Spires of Thin
\subsection{Legend of the Shuffling Spires of Thin}
\label{sec:legendoftheshufflingspiresofthin}
Old tales tell of the mystic Kingdoms of Thin, all of which border in four equally divided parts of a circle - in the Realm of Thin-Things.
These kingdoms were ruled by four power-hungry Kings, each jealous of the next, their fair Queens, brave Jacks and top $10$ Henchmen numbered - you guessed it - one through ten.
Even though the Kings’ respective wealth were great, they were not content with ruling only one of the Kingdoms of Thin. Each wanted to claim the entire Realm for their own. Since all Kingdoms were equally thin, this would require very precise plotting.
...and we all know that plotting means spying, and you cannot spell ‘spyglass’ without ‘spy’, and the only way to use a spyglass effectively is to use it from something very wide, lengthy or high.

Since width is hard to come by in the Realm of Thin-Things, and ‘length’ is a more bothersome word, the Kings all plotted to construct high Spires, so they could spy far-away from their Kingdom - for the right moment when they could claim the Realm for themselves.

The problem with spires, though, is that they’re immobile - and once you’ve placed a spire it may be hard to move it if one would wish to spy somewhere else.
After all, to dominate the Realm of Thin-Things, one would have to spy on all four Kingdoms!

Luckily, in the Lands of Thin, one may construct remarkably thin-things.
So the Kings plotted to build four equally high and thin spires, all constructed in three very - even in the Realm of Thin-Things - thin sections.
The low parts were assigned to the Jacks, who were the most suitable 'Thins' to guard and protect the towers, the middle bits were assigned to the Queens and the highest sections were given unto the Kings themselves.

These sections were so amazingly thin that the wind would easily carry them anywhere across the Lands of Thin. This way; the Kings could spy on all of their neighbours in wait for the right moment to strike.

When this opportunity had come it was decided that the Kingdoms’ ten best Henchmen were to launch an all-out assault targeted at the other Kingdoms.
This would be the way to a unified Realm of Thin-Things, thought the Kings.

And so one day a mighty storm swept the Kingdoms of Thin and took even the least-thin Henchmen aloft.
The Realm was veiled in powerful winds, the likes of which neither Kingdom had ever seen. In the midst of all this chaos, all four Kings decided to put their respective master plan to work and gaze from their high Spires as they conquered the Lands of Thin.

But it came to pass that the Kings of Thin were also men of thin intellect, as it did not occur to them that their high Spires had also been swept from the lands and were being taken far away from their Kingdom. The High Spires of Thin were swept across the Realm, across the Kingdoms of Thin and shuffled so that neither King knew what Kingdom nor Spire he was in.
Yet neither side's Henchmen saw any difference.

The Kings and their Shuffling Spires were about to be attacked by not only the other armies, but by their own.
Thus, the Legend of the Shuffling Spires of Thin was made.

\textbf{Shuffle-Spires} is based around the four Kings, and the Legendary Spires of Thin.
% ss_setup.tex

% Setting up the board
\subsection{Setting up the board}
\label{sec:settinguptheboard}
Like most things in the Realm of Thin-Things, there is not much room for humour.
Therefore, the joker has no place in either Kingdom; remove all jokers from the deck.

The \textbf{Shuffle-Spires} board is set up using the Kings, Queens and Jacks (of all suits) to represent the four Spires and their three sections.
One could imagine the Spires stretching outward from the center.
Shuffle the rest of the deck, containing cards one-through-ten (Ace being nr. 1) from all suits, and place it in the middle of the board.
The Spires thusly create three different zones on the board, which will represent progression of the attacking Henchmen.

Each player chooses a King (a suit, rather), and places him-/herself next to the currently inhabited Spire.
Note that the player is not the owner of any specific Spire, but rather three parts that will change location during play.

The King of Hearts starts off with the title \textit{'The Thin-Things-Thin'}, to signify him being the current Ruler of the Realm of Thin-Things.
The title dictates when the Henchmen are moved further towards the center of the Spires.
This title may be given unto someone else during play, which will be explained in latter parts of this ruleset.

\begin{figure}[h!]
	\centering
	\includegraphics[width=\linewidth]{img/starting.png}
	\caption{Starting board and progression zones.}
	\label{fig:starting}
\end{figure}

\paragraph{Note}
Since the Spires may shuffle and swap during play, in this ruleset, \textit{'X's Spire'} will signify the Spire Player X's King currently resides in.

\input{ss_howto.tex}
% ss_towin.tex

\subsection{Winning \textbf{Shuffle-Spires}}
\label{sec:winningshufflespires}
\textbf{Shuffle-Spires} is a game about \textit{attempting} to predict the chaos the storm and winds have wrought upon the Kings of Thin, their Legendary Spires and their plan for Realm-Domination.

The objective in \textbf{Shuffle-Spires} is to be the sole surviving King of Thin (and thereof the ruler of the Realm of Thin-Things).
That is; to keep one’s top section (King) from being destroyed for as long as possible.


\subsection{About}
\label{sec:about}
The playing card models used to compile figures \ref{fig:starting} to \ref{fig:shuffle} are created by Colin M.L. Burnett and licensed under the \href{http://en.wikipedia.org/wiki/en:Creative_Commons}{Creative Commons} \href{http://creativecommons.org/licenses/by-sa/3.0/deed.en}{Attribution-Share Alike 3.0 Unported} license.
\textbf{Shuffle-Spires} itself is licensed under the \href{http://en.wikipedia.org/wiki/en:Creative_Commons}{Creative Commons} \href{http://creativecommons.org/licenses/by/4.0/}{Attribution 4.0} license.\\

\noindent
I wish to thank those who helped in playtesting.
Feel free to contact me if there are any ideas as to how one may improve the experience.\\

\noindent
Eric Nilsson \\
\href{mailto:CaterHatterPillar@gmail.com}{CaterHatterPillar@gmail.com}

\end{document}
